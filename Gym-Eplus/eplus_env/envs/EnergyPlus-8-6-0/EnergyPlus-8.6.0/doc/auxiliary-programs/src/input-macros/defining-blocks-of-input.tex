\section{Defining Blocks of Input}\label{defining-blocks-of-input}

The \textbf{\#\#def} command allows a block of input text to be defined and given a name. The block of text can then be inserted anywhere in the EnergyPlus input stream by simply referencing the name of the block. (This process is called macro expansion.) The block can have parameters (also called arguments) that can be given different values each time the block is referenced.

The syntax of the \textbf{\#\#def} command is as follows:

\begin{lstlisting}
        unique name
           |         zero or more arguments   macro text
           |            |                     |
           |            |                     |
    ##def  macro-name [ arg1 arg2,arg3 ...]   text line 1
  |_     |           | |    |_   |   _|     |
    |    |           | |      |  |  |       |
    |    |           | |      |  |  |       
zero    one       zero_|     space(s)       one
or      or        or         or comma       or
more    more      more                      more
spaces  spaces    spaces                    spaces
\end{lstlisting}

Example: Define a schedule macro with name ``All\_Const'':

\begin{lstlisting}
**##def** All_Const[x]
**       ** Fraction, WeekON, 1,1, 12,31;
    WEEKSCHEDULE, WeekON,
        DayON,DayON,DayON,
        DayON,DayON,DayON,
        DayON,DayON,DayON,
        DayON,DayON,DayON;
    DAYSCHEDULE, DayON, Fraction,
        x,x,x,x,x,x,x,x,x,x,x,x,
        x,x,x,x,x,x,x,x,x,x,x,x ;
**##enddef**
\end{lstlisting}

Then, in the EnergyPlus input stream (file in.imf), when we write :

\begin{lstlisting}

SCHEDULE, Constant At 0.8, All_Const[0.8]
\end{lstlisting}

the result (file out.idf) is equivalent to:

\begin{lstlisting}
SCHEDULE, Constant At 0.8,
        Fraction, WeekON, 1,1, 12,31;
    WEEKSCHEDULE, WeekON,
        DayON,DayON,DayON,
        DayON,DayON,DayON,
        DayON,DayON,DayON,
        DayON,DayON,DayON;
    DAYSCHEDULE, DayON, Fraction,
        0.8,0.8,0.8,0.8,0.8,0.8,0.8,0.8,0.8,0.8,0.8,0.8,
        0.8,0.8,0.8,0.8,0.8,0.8,0.8,0.8,0.8,0.8,0.8,0.8 ;
\end{lstlisting}

Macro definitions may have one or more arguments; the maximum number of arguments is 32. When a macro with arguments is referenced, its arguments must be given values. When a macro has no arguments, the brackets are still required both for macro definition and reference.

Caution: Square brackets {[} {]} have been used in some versions of EnergyPlus inputs as comment/units fields. These will be expanded if left in the IDF and sent to EPMacro.

Macro names must be unique (except see \textbf{\#\#set1} below); i.e., when a macro name is defined it cannot be defined again. Macro names are limited to 40 characters.

To summarize, commands you use to define macros are the following:

\textbf{\#\#def} macro-name {[}\textbf{arg1},..,\textbf{arg\emph{n}} {]} macro-text

Defines a macro with the name macro-name and arguments ``arg1'' through ``arg\emph{n}''. ``Macro-text'' is one or more lines of text. If there are no arguments, the syntax is \textbf{\#\#def} macro-name macro-text.

\textbf{\#\#enddef}

Indicates the end of the macro definition initiated by \textbf{\#\#def.}

\textbf{\#\#def1} macro-name {[}\textbf{arg1},..,\textbf{arg\emph{n}} {]} macro-text

This is the same as \textbf{\#\#def} but there is only one line of text so that the terminating command \textbf{\#\#enddef} is not required.

\textbf{\#\#set1} macro-name macro-text

Like \textbf{\#\#def1} but has no arguments and macro-text is evaluated before storing. ``Macro-text is evaluated'' means that if macro-text contains other macros, these macros will be expanded, and the expanded text becomes the macro-text defined by \textbf{\#\#set1}.

Example:

\begin{lstlisting}
                **\#\#def1**   xx  123
                **\#\#set1**   yy  xx[]
\end{lstlisting}

is equivalent to: \textbf{\#\#set1} yy 123

\textbf{\#\#set1} can also be used to redefine macro-name.

\textbf{\#\#set1} x 0 . . .

\textbf{\#\#set1} x \textbf{\#eval}{[} x{[}{]}+1 {]}

(see Arithmetic Operations for description of the \textbf{\#eval} macro.)

\subsection{Arithmetic Operations}\label{arithmetic-operations}

The built-in macro called \textbf{\#eval{[} {]}} can be used to perform arithmetic, literal, and logical operations. It can be abbreviated to \textbf{\# {[} {]}.}

\#eval{[} X OP Y {]} or \#{[} X OP Y {]}

gives the result X OP Y. The allowed values for X, OP, and Y, and the corresponding result, are shown in the following table.

\begin{longtable}[c]{p{1.5in}p{1.5in}p{1.5in}p{1.5in}}
\toprule 
X* & OP ** & Y & Result \tabularnewline
\midrule
\endfirsthead

\toprule 
X* & OP ** & Y & Result \tabularnewline
\midrule
\endhead

number & + (plus) & number & number \tabularnewline
number & - (minus) & number & number \tabularnewline
number & * (times) & number & number \tabularnewline
number & / (divided by) & number & number \tabularnewline
number & min & number & number \tabularnewline
number & max & number & number \tabularnewline
number & mod & number & number \tabularnewline
number & ** (power) & number & number \tabularnewline
SIN & OF & number (degrees) & number \tabularnewline
COS & OF & number (degrees) & number \tabularnewline
TAN & OF & number (degrees) & number \tabularnewline
SQRT & OF & number & number \tabularnewline
ABS & OF & number & number \tabularnewline
ASIN & OF & number & number (degrees) \tabularnewline
ACOS & OF & number & number (degrees) \tabularnewline
ATAN & OF & number & number \tabularnewline
INT & OF & number & number \tabularnewline
LOG10 & OF & number & number \tabularnewline
LOG & OF & number & number \tabularnewline
literal1 & // (concatenate) & literal2 & literal "literal1literal2" \tabularnewline
literal1 & /// (concatenate) & literal2 & literal "literal1 literal2" \tabularnewline
literal & EQS (=) & literal & logical (true or false) case sensitive \tabularnewline
literal & NES (\(\neq\)) & literal & logical (true or false) case sensitive \tabularnewline
literal & EQSU (=) & literal & logical (true or false) not case sensitive \tabularnewline
literal & NESU (\(\neq\)) & literal & logical (true or false) not case sensitive \tabularnewline
logical & AND & logical & logical (true or false) \tabularnewline
logical & OR & logical & logical (true or false) \tabularnewline
 & NOT & logical & logical (true or false) \tabularnewline
number & EQ (=) & number & logical (true or false) \tabularnewline
number & NE (\(\neq\)) & number & logical (true or false) \tabularnewline
number & GT (>) & number & logical (true or false) \tabularnewline
number & GE (\(\geq\)) & number & logical (true or false) \tabularnewline
number & LT (<) & number & logical (true or false) \tabularnewline
number & LE (\(\leq\)) & number & logical (true or false) \tabularnewline
* Upper or lower case is allowed for SIN, COS, etc. \tabularnewline
** Upper or lower case is allowed for OF, EQS, etc. \tabularnewline
\bottomrule
\end{longtable}

Example

\textbf{\#eval{[}** 1 + 2 **{]}} when expanded becomes 3.

\textbf{\#eval{[}** 1 +}\#eval\textbf{{[}2 * 3{]} **{]}} when expanded becomes 7.

Example

\textbf{\#\#set1} city{[}{]} Washington

DesignDay, \#{[} city{[} {]} /// SUMMER {]}, ! Design Day Name

gives

\begin{lstlisting}
 DesignDay, "Washington SUMMER", ! Design Day Name
\end{lstlisting}

The following example illustrates the use of \textbf{\#eval} inside \textbf{\#if} commands:

\textbf{\#\#if} \textbf{\#{[}** city{[} {]} EQS Chicago **{]}}

\textbf{\#\#if} \textbf{\#{[}\#{[}** city{[} {]} EQS Chicago {]} and}\#{[}** occup{[} {]} NES low **{]}** **{]}**

Notes:

\begin{enumerate}
\def\labelenumi{\arabic{enumi}.}
\tightlist
\item
  For logical values:
\end{enumerate}

False = 0 or BLANK,

True = any other character

\begin{enumerate}
\def\labelenumi{\arabic{enumi}.}
\setcounter{enumi}{1}
\tightlist
\item
  A literal must be enclosed inside a pair of double quotes if it contains BLANKs or reserved characters like {[} {]} ( ) ,
\end{enumerate}

E.g., ``abc *def''

Otherwise, the quotes around the literals are optional.

\begin{enumerate}
\def\labelenumi{\arabic{enumi}.}
\setcounter{enumi}{2}
\tightlist
\item
  Literal concatenation operators // and /// produce quoted literals.
\end{enumerate}

E.g., \# {[} large /// office {]} gives ``large office''

\begin{enumerate}
\def\labelenumi{\arabic{enumi}.}
\setcounter{enumi}{3}
\item
  Literals are case sensitive. For example, ``Chicago'', ``CHICAGO'' and ``chicago'' are distinct.
\item
  EQS and NES are case sensitive string comparisons. EQSU and NESU are case insensitive string comparisons.
\item
  Literals are limited to 40 characters.
\end{enumerate}
