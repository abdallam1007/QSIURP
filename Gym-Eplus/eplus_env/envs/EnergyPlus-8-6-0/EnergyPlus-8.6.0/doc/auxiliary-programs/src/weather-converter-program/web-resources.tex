\section{Web Resources}\label{web-resources}

Building Energy Tools Directory, a directory of information on 340+ energy tools from around the world.

\url{http://www.energytoolsdirectory.gov/}

Energy Systems Research Unit, University of Strathclyde, authors of ESP-r, up-to-date information on ESP-r and other energy systems research and software development.

\url{http://www.strath.ac.uk/Departments/ESRU}

EnergyPlus, up-to-date information on the current status of EnergyPlus and working with the team, and documentation such as input data structure, output data structure, and licensing opportunities. Additional weather files may be posted here as well.

\url{http://www.energyplus.gov}

Description of the SWERA project. \url{http://swera.unep.net/swera/}

Weather Analytics (\href{http://www.wxaglobal.com}{www.wxaglobal.com}) - Site specific weather files in EnergyPlus format based on the latest 30 years of hourly data are now available from the private sector company Weather Analytics for any official weather station or over 600,000 35-km grid tiles across the globe. These files are built by integrating hourly weather station observations and the new NOAA reanalysis data sets. Both Typical Meteorological Year (TMY) files and individual, Actual Meteorological Year (AMY) files are available as well as files constructed from the previous 12 months.

Meteonorm (\href{http://www.meteonorm.com}{www.meteonorm.com}) - Files for specific locations can be purchased in EnergyPlus format from Meteonorm company. Meteonorm extrapolates hourly data from statistical data for a location. Where statistical data aren't available, Meteonorm interpolates from other nearby sites. Generally a statistical approach is a last resort--weather files generated from statistics will not demonstrate the normal hour-to-hour and day-to-day variability seen in measured data.
