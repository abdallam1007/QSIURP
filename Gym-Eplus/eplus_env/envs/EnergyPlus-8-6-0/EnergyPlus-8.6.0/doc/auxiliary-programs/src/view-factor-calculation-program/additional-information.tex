\section{Additional Information}\label{additional-information}

The interface and this document do not describe all of the features of the View3D program. Additional information can be found in the NIST document View3D32.pdf that accompanies the distribution. For example, if an obstruction wall is desired, it can be placed using the interface, but then the ``C'' at the left end of the row describing that surface in the input file should be changed to ``O''. The program can then be rerun with the new input file. If View3D.exe is double clicked, it will ask for the names of the input file and the output file.

An additional point should be emphasized. The program will not calculate view factors for walls containing windows. That is all surfaces must be convex. Therefore, a wall containing a subsurface must be described as four sections surrounding the subsurface. They can be combined using the ``comb'' column as described in the View3D document. However, this in not necessary if the user is willing to work with a few additional surfaces.
