\section{Example 10. Plant Loop Override Control}\label{example-10.-plant-loop-override-control}

\subsection{Problem Statement}\label{problem-statement-000}

A common occurrence in EnergyPlus central plant simulations is for a component to be designed well, but during the course of an annual simulation, it is operated outside of its allowable region.~ This is due to the governing control strategies (operating schemes).~ These operation schemes may not have the intelligence to say, turn off a cooling tower when the outdoor temperature is too low.

Within the EnergyPlus example files, the cooling tower offers warnings stating that the tower temperature is going below a low temperature limit.~ We should ask, can we use a simple EMS addition to an input file to override the loop and turn off the cooling tower to avoid these situations and therefore the warnings?

\subsection{EMS Design Discussion}\label{ems-design-discussion-000}

For this example, we will start with the example file that is packaged with EnergyPlus called EcoRoofOrlando.idf.~ This is one example of an input file where a cooling tower throws warnings due to operating at too low of a temperature.~ Although the input file has many aspects related to zone and HVAC, we will only be interested in the loop containing the tower, which is a \emph{CondenserLoop} named \emph{Chiller Plant Condenser Loop.} The loop has a minimum loop temperature of 5 degrees Celsius, as specified by the \emph{CondenserLoop} object.

In order to avoid these warnings and properly shut off the tower, EMS will be used to check the outdoor temperature and shut down the whole loop.~ Special care must be taken when manipulating plant and condenser loops with EMS.~ The most robust way found is to both disable the loop equipment and also override (turn off) the loop.~ Skipping either of these can cause mismatches where either components are still expecting flow but the pump is not running, or the pump is trying to force flow through components which are disabled.~ Either of these cases can cause unstable conditions and possibly fatal flow errors.

The outdoor air temperature must be checked in order to determine what the EMS needs to do at a particular point in the simulation.~ This is handled by use of an EMS sensor monitoring the Outdoor Dry Bulb standard E+ output variable.

To manage the loop and pump, actuators are employed on both.~ The pump actuator is a mass flow rate override.~ This can be used to set the flow to zero, effectively shutting off the pump.~ The loop actuator is an on/off supervisory control, which allows you to ``shut the entire loop down.''~ This actuator will not actually shut the loop down, it effectively empties the equipment availability list, so that there is no equipment available to reject/absorb load on the supply side.~ If you use this actuator alone to ``shut down the loop,'' you may find that the pump is still flowing fluid around the loop, but the equipment will not be running.

The EMS calling point chosen is ``\emph{InsideHVACSystemIterationLoop,}'' so that the operation will be updated every time the plant loops are simulated.

The Erl program is quite simple for this case.~ If the outdoor dry bulb temperature goes below a certain value, the loop and pump actuators are set to zero.~ If the outdoor temperature is equal to or above this value, the actuators are set to \emph{Null}, relinquishing control back to the regular operation schemes.~ In modifying this specific input file it was found that the outdoor dry bulb temperature which removed these warnings was six degrees Celsius.~ We also create a custom output variable called ``EMS Condenser Flow Override On'' to easily record when the overrides have occurred.

\subsection{EMS Input Objects}\label{ems-input-objects-000}

The main input objects that implement this example of plant loop control are listed below and are included in the example file called ``EMSPlantLoopOverrideControl.idf.''~ The addition of the EMS objects properly shuts down the loop as the outdoor temperature go below the transition value, and the simulation error file shows no warnings for the tower outlet temperature.

\begin{lstlisting}

  EnergyManagementSystem:Sensor,
      OutdoorTemp,         !- Name
      Environment,         !- Output:Variable Index Key Name
      Site Outdoor Air Drybulb Temperature;    !- Output:Variable Name


    EnergyManagementSystem:Actuator,
      Actuator_Loop,       !- Name
      Chiller Plant Condenser Loop, !- Actuated Component Unique Name
      Plant Loop Overall,  !- Actuated Component Type
      On/Off Supervisory;  !- Actuated Component Control Type


    EnergyManagementSystem:Actuator,
      PumpFlowOverride,    !- Name
      Chiller Plant Cnd Circ Pump,  !- Actuated Component Unique Name
      Pump,                !- Actuated Component Type
      Pump Mass Flow Rate; !- Actuated Component Control Type


    EnergyManagementSystem:GlobalVariable,
      PumpFlowOverrideReport;


    EnergyManagementSystem:OutputVariable,
      EMS Condenser Flow Override On [On/Off], !- Name
      PumpFlowOverrideReport,    !- EMS Variable Name
      Averaged,            !- Type of Data in Variable
      SystemTimeStep;      !- Update Frequency






    EnergyManagementSystem:ProgramCallingManager,
      Condenser OnOff Management,
      InsideHVACSystemIterationLoop,
      TowerControl;


    EnergyManagementSystem:Program,
      TowerControl,
      IF (OutdoorTemp < 6.0),
        SET Actuator_Loop = 0.0,
        SET PumpFlowOverride = 0.0,
        SET PumpFlowOverrideReport = 1.0,
      ELSE,
        SET Actuator_Loop = Null,
        SET PumpFlowOverride = Null,
        SET PumpFlowOverrideReport = 0.0,
      ENDIF;


    Output:Variable,
      *,
      EMS Condenser Flow Override On,
      Hourly;
\end{lstlisting}
