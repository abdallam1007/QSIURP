\section{Zone Forced Air Unit}\label{zone-forced-air-unit}

The input object called ZoneHVAC:ForcedAir:UserDefined provides a shell for creating custom models of a device that serves as a single-zone HVAC unit that operates by circulating air in and out of the zone.~ This device is analogous to those component models in the Group -- Zone Forced Air Units, such as ZoneHVAC:PackagedTerminalAirConditioner or ZoneHVAC:WaterToAirHeatPump.

In addition to the primary air connection that connects to the zone, there are options for additional connections to a second air stream (e.g.~for outdoor ventilation or heat source or sink), up to three separate plant loop connections (e.g.~hot water, chilled water, heat rejection), a water supply tank, a water collection tank, and a separate zone for skin losses.

The zone unit is associated with a thermal zone (by the ZoneHVAC:EquipmentConnections and ZoneHVAC:EquipmentList objects).~ In EnergyPlus, when there are controlled thermal zones with thermostat (and humidistat) controls, the central routines predict the loads that zone equipment need to meet in order to maintain control of the zone conditions.~ When there are multiple types of equipment serving a zone, they are sequenced to meet heating or cooling loads in a particular order.~ Rather than the total predicted load, the second or third devices need to know the load that remains after the earlier-sequenced devices have already operated on the zone.~ The following internal variables are useful inputs for controlling zone equipment in your models:

\begin{itemize}
\item
  An internal variable called ``Remaining Sensible Load to Heating Setpoint'' provides the current value for the sensible load, in {[}W{]}, that remains for this device that if delivered will allow the zone to reach the heating setpoint under current conditions.
\item
  An internal variable called ``Remaining Sensible Load to Cooling Setpoint'' provides the current value for the sensible load, in {[}W{]}, that remains for this device that if delivered will allow the zone to reach the cooling setpoint under current conditions.
\item
  An internal variable called ``Remaining Latent Load to Humidifying Setpoint'' provides the current value for the latent load, in {[}kg/s{]}, that remains for this device that if delivered will allow the zone to reach the humidification setpoint under current conditions.
\item
  An internal variable called ``Remaining Latent Load to Dehumidifying Setpoint'' provides the current value for the latent load, in {[}kg/s{]}, that remains for this device that if delivered will allow the zone to reach the dehumidification setpoint under current conditions.
\end{itemize}

\subsection{Primary Air Connection}\label{primary-air-connection-000}

The primary air connection includes both an inlet and an outlet that are required to be used when using this component.~ This is called the primary air connection because it is how the zone unit is connected to the zone.~ The inlet to the custom zone unit is a node that is also an exhaust outlet from the zone.~ The following EMS internal variables are made available for this inlet node and should be useful inputs to your own custom models:

\begin{itemize}
\item
  An internal variable called ``Inlet Temperature for Primary Air Connection,'' provides the current value for the drybulb air temperature at the component's inlet node, in {[}C{]}.
\item
  An internal variable called ``Inlet Humidity Ratio for Primary Air Connection,'' provides the current value for the moist air humidity ratio at the component's inlet node, in {[}kgWater/kgDryAir{]}
\item
  An internal variable called ``Inlet Density for Primary Air Connection,'' provides the current value for the density of moist air at the component's main inlet node, in {[}kg/m\(^{3}\){]}.
\item
  An internal variable called ``Inlet Specific Heat for Primary Air Connection,'' provides the current value for the specific heat of moist air at the component's main inlet node, in {[}J/kg-C{]}.
\end{itemize}

The inlet node also has an actuator associated with it so that the rate of air flow leaving the thermal zone and entering the unit can be passed to the rest of EnergyPlus.

\begin{itemize}
\tightlist
\item
  An actuator called ``Primary Air Connection,'' with the control type ``Inlet Mass Flow Rate,'' in {[}kg/s{]}, needs to be used.~ This will set the flow rate of air leaving the zone through the zone exhaust air node.
\end{itemize}

The primary outlet for the custom zone unit is a node that is also an inlet to the zone.~ The following EMS actuators are created for this outlet node and must be used to pass results from the custom model to the rest of EnergyPlus:

\begin{itemize}
\item
  An actuator called ``Primary Air Connection,'' with the control type ``Outlet Temperature,'' in {[}C{]}, needs to be used.~ This will set the drybulb temperature of the air leaving the zone unit and entering the zone through the zone air inlet node.
\item
  An actuator called ``Primary Air Connection,'' with the control type ``Outlet Humidity Ratio,'' in {[}kgWater/kgDryAir{]}, needs to be used.~ This will set the humidity ratio of the air leaving the zone unit and entering the zone through the zone air inlet node.
\item
  An actuator called ``Primary Air Connection,'' with the control type ``Outlet Mass Flow Rate,'' in {[}kg/s{]}, needs to be used.~ This will set the flow rate of air leaving the zone unit and entering the zone through the zone air inlet node.
\end{itemize}

It is not required that the primary air connections inlet and outlet mass flow rates be identical.~ However, if there is an imbalance, then the model should use the secondary air connection to balance air mass flows.

\subsection{Secondary Air Connection}\label{secondary-air-connection-000}

The secondary air connection provides options for an added inlet node, or outlet node, or both depending on the user's needs.~ This separate air stream can be used for outdoor air ventilation or as a source or sink for energy.~ The secondary air inlet node will often be defined to be an outdoor air node (ref. OutdoorAir:Node) but that is not required.~ The secondary air outlet node can be used as relief exhaust when the unit is providing outdoor air ventilation.~ If the secondary air outlet is not really connected to anything else and just releases air to the outdoors, then it isn't necessary that moist air properties be set using actuators because they will not impact anything else in the model.

If the secondary air connection inlet node is used, then the following internal variables and actuator are made available:

\begin{itemize}
\item
  An internal variable called ``Inlet Temperature for Secondary Air Connection,'' provides the current value for the drybulb air temperature at the secondary inlet node, in {[}C{]}.
\item
  An internal variable called ``Inlet Humidity Ratio for Secondary Air Connection,'' provides the current value for the moist air humidity ratio at the secondary inlet node, in {[}kgWater/kgDryAir{]}
\item
  An internal variable called ``Inlet Density for Secondary Air Connection,'' provides the current value for the density of moist air at the secondary inlet node, in {[}kg/m\(^{3}\){]}.
\item
  An internal variable called ``Inlet Specific Heat for Secondary Air Connection,'' provides the current value for the specific heat of moist air at the secondary inlet node, in {[}J/kg-C{]}.
\item
  An actuator called ``Secondary Air Connection,'' with the control type ``Inlet Mass Flow Rate,'' in {[}kg/s{]}, needs to be used.~ This will set the flow rate of air entering the zone unit through the secondary air connection inlet.
\end{itemize}

If the secondary air connection outlet node is used, then the following actuators are created:

\begin{itemize}
\item
  An actuator called ``Secondary Air Connection,'' with the control type ``Outlet Temperature,'' in {[}C{]}, needs to be used.~ This will set the drybulb temperature of the air leaving the zone unit through the secondary air outlet node.
\item
  An actuator called ``Secondary Air Connection,'' with the control type ``Outlet Humidity Ratio,'' in {[}kgWater/kgDryAir{]}, needs to be used.~ This will set the humidity ratio of the air leaving the zone unit through the secondary air outlet node.
\item
  An actuator called ``Secondary Air Connection,'' with the control type ``Outlet Mass Flow Rate,'' in {[}kg/s{]}, needs to be used.~ This will set the flow rate of air leaving the zone unit through the secondary air outlet node.
\end{itemize}

\subsection{Plant Connections}\label{plant-connections-002}

The user defined zone unit can also be connected to up to three different plants to provide hydronic-based cooling, heating, and/or heat source or rejection.

Although the zone unit actively conditions the zone, from the point of view of plant they are demand components.~ These plant connections are always ``demand'' in the sense that the zone unit will place loads onto the plant loops serving it and are not configured to be able to meet plant loads in the way that supply equipment could (loading mode is always DemandsLoad).~ These plant connections are always of the type that when flow is requested, the loop will be operated to try and meet the flow request and if not already running, these flow requests can turn on the loop (loop flow request mode is always NeedsFlowAndTurnsLoopOn).

For plant loops, both the inlet and outlet nodes need to be used for each loop connection.~ The ZoneHVAC:ForcedAir:UserDefined object appears directly on the Branch object used to describe the plant.~ The central plant routines require that each plant component be properly initialized and registered. Special actuators are provided for these initializations and they should be filled with values by the Erl programs that are called by the program calling manager assigned to the zone unit for model setup and sizing.~ The following three actuators are created for each of ``\emph{N}'' plant loops and must be used to properly register the plant connection:

\begin{itemize}
\item
  An actuator called ``Plant Connection \emph{N}'' with the control type ``Minimum Mass Flow Rate,'' in {[}kg/s{]}, should be used.~ This will set the so-called hardware limit for component's minimum mass flow rate when operating.~ (If not used, then the limit will be set to zero which may be okay for many if not most models.)
\item
  An actuator called ``Plant Connection \emph{N}'' with the control type ``Maximum Mass Flow Rate,'' in {[}kg/s{]}, needs to be used.~ This will set the so-called hardware limit for the component's maximum mass flow rate when operating.
\item
  An actuator called ``Plant Connection \emph{N}'' with the control type ``Design Volume Flow Rate,'' in {[}m\(^{3}\)/s{]}, needs to be used.~ This will register the size of the component for use in sizing the plant loop and supply equipment that will need to meet the loads.
\end{itemize}

For each plant loop connection that is used, the following internal variables are available for inputs to the custom component model:

\begin{itemize}
\item
  An internal variable called ``Inlet Temperature for Plant Connection \emph{N}'' provides the current value for the temperature of the fluid entering the component, in {[}C{]}.
\item
  An internal variable called ``Inlet Mass Flow Rate for Plant Connection \emph{N}'' provides the current value for the mass flow rate of the fluid entering the component, in {[}kg/s{]}.
\item
  An internal variable called ``Inlet Density for Plant Connection \emph{N}'' provides the current value for the density of the fluid entering the component, in {[}kg/m\(^{3}\){]}.~ This density is sensitive to the fluid type (e.g.~if using glycol in the plant loop) and fluid temperature at the inlet.
\item
  An internal variable called ``Inlet Specific Heat for Plant Connection \emph{N}'' provides the current value for the specific heat of the fluid entering the component, in {[}J/kg-C{]}. This specific heat is sensitive to the fluid type (e.g.~if using glycol in the plant loop) and fluid temperature at the inlet.
\end{itemize}

For each plant loop connection that is used, the following EMS actuators are created and must be used to pass results from the custom model to the rest of EnergyPlus:

\begin{itemize}
\item
  An actuator called ``Plant Connection \emph{N}'' with the control type ``Outlet Temperature,'' in {[}C{]}, needs to be used.~ This is the temperature of the fluid leaving the zone unit through that particular plant connection.
\item
  An actuator called ``Plant Connection N'' with the control type ``Mass Flow Rate,'' in kg/s, needs to be used. This actuator registers the component model's request for plant fluid flow.~ The actual mass flow rate through the component may be different than requested if the overall loop situation is such that not enough flow is available to meet all the various requests.~ In general, this actuator is used to lodge a request for flow, but the more accurate flow rate will be the internal variable called ``Inlet Mass Flow Rate for Plant Connection \emph{N.}''
\end{itemize}

\subsection{Water Use}\label{water-use-002}

The user defined zone unit can be connected to the water use models in EnergyPlus that allow modeling on-site storage.~ If a supply inlet water storage tank is used, then an actuator called ``Water System'' with the control type ``Supplied Volume Flow Rate,'' in m\(^{3}\)/s, needs to be used.~ This sets up the zone unit as a demand component for that storage tank.~ If a collection outlet water storage tank is used, then an actuator called ``Water System'' with the control type ``Collected Volume Flow Rate,'' in m\(^{3}\)/s, needs to be used.

\subsection{Ambient Zone}\label{ambient-zone-002}

The user defined zone unit can be connected to an ambient zone and provide internal gains to that zone.~ The zone can be different than the one that unit is connected to via the primary air connection if desired.~ This is for ``skin losses'' that the unit might have that result from inefficiencies and other non-ideal behavior.~ When an ambient zone is specified, the following actuators are created that can be used for different types of internal gains to the named zone:

\begin{itemize}
\item
  An actuator called ``Component Zone Internal Gain'' with the control type ``Sensible Heat Gain Rate,'' in {[}W{]}, is available.~ This can be used for purely convective sensible heat gains (or losses) to a zone.
\item
  An actuator called ``Component Zone Internal Gain'' with the control type ``Return Air Heat Gain Rate,'' in {[}W{]}, is available.~ This can be used for purely convective sensible heat gains (or losses) to the return air duct for a zone.
\item
  An actuator called ``Component Zone Internal Gain'' with the control type ``Thermal Radiation Heat Gain Rate,'' in {[}W{]}, is available.~ This can be used for thermal radiation gains (or losses) to a zone.
\item
  An actuator called ``Component Zone Internal Gain'' with the control type ``Latent Heat Gain Rate,' in {[}W{]}, is available.~ This can be used for latent moisture gains (or losses) to a zone.
\item
  An actuator called ``Component Zone Internal Gain'' with the control type ``Return Air Latent Heat Gain Rate,'' in {[}W{]}, is available.~ This can be used for latent moisture gains (or losses) to a the return air duct for a zone.
\item
  An actuator called ``Component Zone Internal Gain'' with the control type ``Carbon Dioxide Gain Rate,'' in {[}m\(^{3}\)/s{]}, is available.~ This can be used for carbon dioxide gains (or losses) to a zone.
\item
  An actuator called ``Component Zone Internal Gain'' with the control type ``Gaseous Contaminant Gain Rate,'' in {[}m\(^{3}\)/s{]}, is available.~ This can be used for generic gaseous air pollutant gains (or losses) to a zone.
\end{itemize}
