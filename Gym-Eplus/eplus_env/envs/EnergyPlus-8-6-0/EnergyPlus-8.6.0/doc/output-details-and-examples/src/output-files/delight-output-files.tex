\section{Delight output files}\label{delight-output-files}

\subsection{eplusout.delightin}\label{eplusout.delightin}

Following completion of an EnergyPlus run that includes Daylighting:Controls objects that use the DElight Daylighting Method, an ASCII text file created during the run is given a file name that consists of the project name appended with DElight.in (e.g., MyProjectDElight.in).

This text file is a formatted DElight input file that was created from the EnergyPlus input data relevant to a DElight simulation. This file can be manually reviewed to determine the exact data that were transformed from EnergyPlus into DElight input.

\subsection{eplusout.delightout}\label{eplusout.delightout}

Following completion of an EnergyPlus run that includes Daylighting:Controls objects that use the DElight Daylighting Method, an ASCII text file created during the run is given a file name that consists of the project name appended with DElight.out (e.g., MyProjectDElight.out).

This text file is a formatted DElight output file that was generated by the DElight simulation engine following the pre-processing daylight factors calculation. This file can be manually reviewed to see the results of these calculations. The file contains an echo of DElight input data, as well as the results of intermediate calculations such as geometrical transformations, surface gridding, and daylight factors, including the following.

\textbf{Surface Data}

\begin{itemize}
\item
  Vertex coordinates in the Building coordinate system. Search for the string ``BldgSystem\_Surface\_Vertices'' within the output file.
\item
  Exterior face luminance values under overcast skies, and for each sun position under clear skies. Search for the string ``Surface Exterior Luminance'' within the output file.
\item
  Common data for radiosity nodal patches for each surface including Area and Number of Nodes. Search for the string ``Surface\_Node\_Area'' within the output file.
\item
  Individual data for radiosity nodal patches for each node on each surface including: building coordinate system coordinates; direct and total luminance values under overcast skies, and for each sun position under clear skies. Search for the string ``BldgSystem\_Node\_Coordinates'' within the output file.
\end{itemize}

\textbf{Reference Point Data}

\begin{itemize}
\item
  Illuminance values from the daylighting factors preprocessor for overcast skies, and for each sun position under clear skies.
\item
  Daylight Factor values from the daylighting factors preprocessor for overcast skies, and for each sun position under clear skies.
\item
  NOTE: The Monthly Average data for Daylight Illuminances and Electric Lighting Reduction will all be zero since these data are not calculated as part of the pre-processing done by the point at which this output file is generated for EnergyPlus.
\end{itemize}
