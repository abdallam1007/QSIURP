\chapter{Running EnergyPlus}\label{running-energyplus}

EnergyPlus is written in C++ and runs as a console application. More explicit details on running EnergyPlus are available in a separate document (Running EnergyPlus in Auxiliary Programs document). Optional command-line arguments are available (energyplus --help, or man energyplus on Linux systems). The following files are used to run EnergyPlus:

\begin{itemize}
\item
  EnergyPlus.exe (the executable file)
\item
  Energy+.ini (described below)
\item
  Energy+.idd (the input data dictionary file)
\item
  In.idf (the input file)
\item
  In.epw -- optional (weather data file)
\end{itemize}

The input data dictionary and input data file have been discussed in the previous sections of this document.

For weather simulations, EnergyPlus accepts EnergyPlus weather files. Previous versions accepted BLAST formatted weather files and now a BLASTWeatherConverter program is provided.~ The actual file name is \textbf{in.epw}.

The Energy+.ini file is a ``standard'' Windows™ ini file and can be manipulated using the Windows API calls though EnergyPlus uses standard Fortran to manipulate it.~ It is a very simple ini file and is fully described in the \href{file:///E:/Docs4PDFs/AuxiliaryPrograms.pdf}{Auxiliary Programs} document. Energy+.ini and in.idf file should be in the directory from which you are running EnergyPlus.exe.

For the advanced user, there is also the ``EPMacro'' program, described in the Auxiliary Programs Document.~ You run it as a separate program before EnergyPlus (the batch file included in the install and shown in the GettingStarted document contains the commands).

EnergyPlus creates the following files (plus more):

% table 2
\begin{longtable}[c]{@{}ll@{}}
\caption{EnergyPlus Output Files \label{table:energyplus-output-files}} \tabularnewline
\toprule 
FileName & Description \tabularnewline
\midrule
\endfirsthead

\caption[]{EnergyPlus Output Files} \tabularnewline
\toprule 
FileName & Description \tabularnewline
\midrule
\endhead

Audit.out & Echo of input \tabularnewline
Eplusout.err & Error file \tabularnewline
Eplusout.eso & Standard Output File \tabularnewline
Eplusout.eio & One time output file \tabularnewline
Eplusout.rdd & Report Variable Data Dictionary \tabularnewline
Eplusout.dxf & DXF (from Report,Surfaces,DXF;) \tabularnewline
Eplusout.end & A one line summary of success or failure \tabularnewline
\bottomrule
\end{longtable}

The eplusout.err file may contain three levels of errors (Warning, Severe, Fatal) as well as the possibility of just message lines.~ These errors may be duplicated in other files (such as the standard output file).

% table 3
\begin{longtable}[c]{@{}ll@{}}
\caption{EnergyPlus Errors \label{table:energyplus-errors}} \tabularnewline
\toprule 
Error Level & Action \tabularnewline
\midrule
\endfirsthead

\caption[]{EnergyPlus Errors} \tabularnewline
\toprule 
Error Level & Action \tabularnewline
\midrule
\endhead

Warning & Take note \tabularnewline
Severe & Should Fix \tabularnewline
Fatal & Program will abort \tabularnewline
\bottomrule
\end{longtable}

EnergyPlus produces several messages as it is executing, as a guide to its progress.~ For example, the run of the 1ZoneUncontrolled input file from Appendix A produces:

\begin{lstlisting}

EnergyPlus Starting
   EnergyPlus 1.3.0.011, 4/5/2006 2:59 PM
   Initializing New Environment Parameters
   Warming up {1}
   Initializing Response Factors
   Initializing Window Optical Properties
   Initializing Solar Calculations
   Initializing HVAC
   Warming up {2}
   Warming up {3}
   Warming up {4}
   Starting Simulation at 12/21 for DENVER_STAPLETON ANN HTG 99% CONDNS DB
   Initializing New Environment Parameters
   Warming up {1}
   Warming up {2}
   Warming up {3}
   Starting Simulation at 07/21 for DENVER_STAPLETON ANN CLG 1% CONDNS DB = >MWB
   EnergyPlus Run Time = 00hr 00min  1.00sec
\end{lstlisting}

Extensive timing studies and fine-tuning of EnergyPlus is NOT complete.~ To give you an idea of comparable run times, we present the following (does not include HVAC) with an early version of EnergyPlus running on a 450MHZ machine.~ Remember, BLAST would be 1 calculation per hour, EnergyPlus (in this case) was 4 calculations per hour.~ Obviously, these are quite out of date.~ However, a recent change in a developer's test machine illustrates the importance of maximum memory.~ A 5 zone full year run on a 1.8GHZ, 1GB machine was running about 8 minutes -- with a new 2.1GHZ, 2GB machine the same file takes about 2 minutes.

% table 4
\begin{longtable}[c]{p{3.0in}p{1.5in}p{1.5in}}
\caption{Timings Comparison (EnergyPlus vs. BLAST) \label{table:timings-comparison-energyplus-vs.-blast}} \tabularnewline
\toprule 
File & BLAST Per Zone & EnergyPlus Per Zone \tabularnewline
\midrule
\endfirsthead

\caption[]{Timings Comparison (EnergyPlus vs. BLAST)} \tabularnewline
\toprule 
File & BLAST Per Zone & EnergyPlus Per Zone \tabularnewline
\midrule
\endhead

GeometryTest (5 Zones, 2 Design Day, Full Weather Year) & 13 sec & 33 sec \tabularnewline
SolarShadingTest (9 Zones, Full Weather Year) & 7 sec & 25 sec \tabularnewline
\bottomrule
\end{longtable}
