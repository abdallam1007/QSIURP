\section{Example Error Messages from Module GetInput routines}\label{example-error-messages-from-module-getinput-routines}

As the simulation starts, each module gets called and gets the values from the input file. These are usually referred to as GetInput routines. They add another error check on the inputs that cannot be fully described by the IDD limits plus they are privy to interactions that their object may have to another object.

\subsection{Warning}\label{warning-002}

\begin{lstlisting}
Site:GroundTemperature:BuildingSurface: Some values fall outside the range of 15-25C.
These values may be inappropriate.  Please consult the Input Output Reference for more details.
\end{lstlisting}

Ground temperatures can have a significant influence on buildings. Values outside the range indicated may give you inaccurate simulation temperatures. Consult the Input Output Reference for more details.

\begin{lstlisting}
GetSurfaceData: CAUTION -- Interzone surfaces are usually in different zones
Surface = WALLMASS, Zone = ZONE1
Surface = iz-WALLMASS, Zone = ZONE1
\end{lstlisting}

Conventionally, interzone surfaces separate two zones. However, some advanced users may create them in the same zone for certain heat transfer efficiencies. This warning message alerts you in case that was not your intention.

\begin{lstlisting}
Weather file location will be used rather than entered Location object.
..Location object = ATLANTA
..Weather File Location = Tampa International Ap FL USA TMY3 WMO# = 722110
..due to location differences, Latitude difference = [5.68] degrees, Longitude difference = [1.89] degrees.
..Time Zone difference = [0.0] hour(s), Elevation difference = [98.10] percent, [309.00] meters.
\end{lstlisting}

You have ``attached'' a weather file that contains different location information than your Site:Location object. The program is warning you of this condition.

\begin{lstlisting}
GetPollutionFactorInput: Requested reporting for Carbon Equivalent Pollution, but insufficient information is entered.
\end{lstlisting}

Both ``FuelFactors'' and ``EnvironmentalImpactFactors'' must be entered or the displayed carbon pollution will all be zero.

You have requested reporting for Carbon Equivalent Pollution (output variables) but you have not entered the required FuelFactor and EnvironmentalImpactFactor objects that are necessary to trigger these outputs properly.

\begin{lstlisting}
BuildingSurface:Detailed = "SURF:xyz", Sun Exposure = "SUNEXPOSED".
 ..This surface is not exposed to External Environment.  Sun exposure has no effect.
\end{lstlisting}

The surface has been entered with SunExposed but it is not an exterior/outdoor surface.

\begin{lstlisting}
GetSurfaceData: InterZone Surface Areas do not match as expected and might not satisfy conservation of energy:
   Area = 1.4E-002 in Surface = 319767, Zone = 2PAV_CONDIC_LOJA_D
   Area = 67.0 in Surface = 6C0708, Zone = 3PAV_CONDIC_TEATRO_G
\end{lstlisting}

Interzone surface areas usually should be matching between the two zones.

\begin{lstlisting}
GetSurfaceData: InterZone Surface Azimuths do not match as expected.
   Azimuth = 270.0, Tilt = 90.0, in Surface = 319767, Zone = 2PAV_CONDIC_LOJA_D
   Azimuth = 180.0, Tilt = 90.0, in Surface = 6C0708, Zone = 3PAV_CONDIC_TEATRO_G
..surface class of base surface = Wall
\end{lstlisting}

Interzone surfaces should be opposite each other -- therefore when Azimuth/Facing do not differ by 180 degrees, a warning is shown. Likewise, Tilt angles should be checked here.

\begin{lstlisting}
GetVertices: Floor is upside down! Tilt angle = [0.0], should be near 180, Surface = "ROOM302-FLOOR", in Zone = "ROOM302".
Automatic fix is attempted.
\end{lstlisting}

\begin{lstlisting}
GetVertices: Roof is upside down! Tilt angle = [180.0], should be near 0, Surface = "ROOM302-CEILING", in Zone = "ROOM302".
Automatic fix is attempted.
\end{lstlisting}

In both of these messages, it has been detected that the outward surface normal for the surfaces is not as expected. With not as expected angles, the sun will not be received on these surfaces (typically), so it is something to correct. The program attempts to fix these -- usually caused by entering the vertices backwards (i.e.~clockwise when should have been counter-clockwise or vice versa).

\begin{lstlisting}
GetInternalHeatGains: Zone = "02AO_FCU04_AN" occupant density is extremely high.
Occupant Density = [14] person/m2.
Occupant Density = [7.000E-002] m2/person. Problems in Temperature Out of Bounds may result.
\end{lstlisting}

The Get Internal Heat Gains routine does some checks as far as Design Level (and maximum schedule * Design Level) and compares to density values. Extremely high gains, especially when no exit for the air (i.e.~infiltration, ventilation) can often result in Temperature Out of Bounds errors (see below in Simulation messages) and these can be fatal.

\begin{lstlisting}
GetVertices: Distance between two vertices < .01, possibly coincident. for Surface = 1%PIANOINTERRATO:UFFICI_WALL_3_0_1, in Zone = 1%PIANOINTERRATO:UFFICI
Vertex [2] = (-53.99,5.86,0.50)
Vertex [1] = (-53.99,5.86,0.51)
Dropping Vertex [2].
\end{lstlisting}

The distance between two vertices is very small (.01 meter \textasciitilde{} .4 inches). This distance is too small for shading calculations and the vertex is dropped.

\begin{lstlisting}
CheckConvexity: Surface = "ZN001:ROOF001" is non-convex.
\end{lstlisting}

Shown when DisplayExtraWarnings is on and a surface is not a convex shape. By itself, this is only a warning but see the severe in the next section when it has impact on the calculations.

\subsection{Severe}\label{severe-002}

\begin{lstlisting}
GetSurfaceData: Some Outward Facing angles of subsurfaces differ significantly from base surface.
...use Output:Diagnostics,DisplayExtraWarnings; to show more details on individual surfaces.
\end{lstlisting}

\begin{lstlisting}
GetSurfaceData: Outward facing angle [95.5] of subsurface = "WL2-1" significantly different than
..facing angle [275.5] of base surface = WEST WALL 2 Tilt = 90.0
..surface class of base surface = Wall
\end{lstlisting}

These are two versions of the same message.~ The first is shown when DisplayExtraWarnings is not activated. The second is shown for details on each subsurface that has the error.~ The error is usually that the subsurface vertices have been entered in opposite order (i.e.~clockwise vs counter-clockwise) from the base surface.

\begin{lstlisting}
This building has no thermal mass which can cause an unstable solution.
Use Material object for all opaque material definitions except very light insulation layers.
\end{lstlisting}

You have probably defined all the surfaces in this building with resistive only constructions (i.e.~object Material:NoMass). An unstable solution can result (including crashes).

\begin{lstlisting}
GetVertices: Distance between two vertices < .01, possibly coincident. for Surface = 1%PIANOINTERRATO:UFFICI_WALL_3_0_1, in Zone = 1%PIANOINTERRATO:UFFICI
Vertex [3] = (-44.82,-12.14,0.51)
Vertex [2] = (-44.82,-12.14,0.50)
Cannot Drop Vertex [3].
Number of Surface Sides at minimum.
\end{lstlisting}

The distance between two vertices is very small (.01 meter \textasciitilde{} .4 inches). This distance is too small for shading calculations but the vertex cannot be dropped as that would bring the surface to less than 3 sides. This surface is degenerate and should be removed from your input file.

\begin{lstlisting}
DetermineShadowingCombinations: Surface = "0%VESPAIO:ZONA1\_ROOF\_1\_6\_0" is a receiving surface and is non-convex.
...Shadowing values may be inaccurate. Check .shd report file for more surface shading details
\end{lstlisting}

Receiving surfaces which are not convex shapes will not be calculated correctly with the shadowing routines. You should view the results carefully.

\subsection{Fatal}\label{fatal-002}

Severes in this realm usually lead to Fatals. Preceding conditions lead to termination.
