\section{Air Loops}\label{air-loops}

\subsection{Definition of Air Loop}\label{definition-of-air-loop}

In EnergyPlus an air loop is a central forced air HVAC system. The term ``loop'' is used because in most cases some air is recirculated so that the air system forms a fluid loop. The air loop is just the ``air side'' of a full HVAC system.~ The input objects related to these air loops begin ``AirLoopHVAC.''

For simulation purposes the air loop is divided into 2 parts: the primary air system (representing the supply side of the loop) and the zone equipment (representing the demand side of the loop). The primary air system includes supply and return fans, central heating and cooling coils, outside air economizer, and any other central conditioning equipment and controls. The zone equipment side of the loop contains the air terminal units as well as fan coils, baseboards, window air conditioners, and so forth. Supply and return plenums are also included on the zone equipment side of the loop.

\subsection{Simulation Method}\label{simulation-method}

Simulating a forced air system and its associated zones can be done in a number of ways. EnergyPlus uses algebraic energy and mass balance equations combined with steady state component models. When the zone air and the air system are modeled with algebraic equations (steady state) then -- in cases with recirculated air -- there will be one or more algebraic loops. In other words it is not possible to solve the equations directly; instead iterative methods are needed. Typically a humidity ratio and a mass flow rate will be variables involved in the iterative solution.

In EnergyPlus the zone humidity ratios and temperatures are decoupled from the solution of the air system equations. The zone air is assigned heat and moisture storage capacities and the capacity effects are modeled by 1\(^{st}\) order ordinary differential equations. During each system simulation time step new zone temperatures and humidities are predicted using past values. The zone temperatures and humidities are then held constant during the simulation of the air system (and the plant). Then the zone temperatures and humidity ratios are corrected using results from the system simulation. As a result the usual algebraic loops arising in steady state air system simulations are eliminated from the EnergyPlus system simulation. The zone temperatures, humidity ratios, and heating and cooling demands are known inputs to the system simulation.

The need for iteration can be reintroduced by the need for system control. If system setpoints are fixed, externally determined, or lagged and control is local (sensor located at a component outlet, actuator at a component inlet) then iteration can be confined to the components and the overall air system equations can be solved directly. However these requirements are too restrictive to simulate real systems. System setpoints are held fixed during a system time step. But controller sensors are allowed to be remote from the location of the actuator. Consequently EnergyPlus uses iteration over the entire primary air system in order to converge the system controllers.

\subsection{Component Models}\label{component-models}

EnergyPlus component models are algorithmic with fixed inputs and outputs. They are embodied as Fortran90 subroutines within software modules. For each component several choices of inputs and outputs are possible. There is no one choice that will be most usable and efficient for all system configurations and control schemes. For reasons of consistency and comprehensibility the requirement was imposed that all EnergyPlus models be forward models. That is, the component inputs correspond to the inlet conditions and the outputs to the outlet conditions.~ If another choice of inputs and outputs is needed it is obtained by numerical inversion of the forward model.

\subsection{Iteration Scheme}\label{iteration-scheme}

The primary air system simulation uses successive substitution starting at the return air inlet node and proceeding in the flow direction to the supply air outlet nodes. The iteration proceeds until an individual actuator-controller has converged (the sensed value matches the setpoint within the specified tolerance). The system controllers are simulated in sequence. During this sequence of iterative solutions the air mass flow rates are held constant. The controllers are converged by the method of interval halving. This method was chosen (rather than for instance Newton-Raphson) for its robustness.

\subsection{Determination of Air Mass Flow Rates}\label{determination-of-air-mass-flow-rates}

In most cases the air mass flow rate of the central air system is set by zone equipment downstream of the primary air system. The air terminal unit components with their built in dampers and controllers respond to the zone heating and cooling loads by setting air flow rates at their inlet nodes. These flow rates are passed back upstream to the primary air system, establishing the flow rates in the primary air system branches. These flow rates are held fixed while the primary air system is simulated.
